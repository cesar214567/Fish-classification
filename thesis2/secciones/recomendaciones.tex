\chapter*{\center \Large RECOMENDACIONES} 
\addcontentsline{toc}{section}{\bfseries RECOMENDACIONES} 
\markboth{RECOMENDACIONES}{RECOMENDACIONES} 

Algunas de las limitaciones que se tiene con este modelo consisten en que el 
\textit{pipeline} depende casi completamente del detector escogido, de tal 
manera que se necesita poder obtener uno bastante robusto y entrenado de 
manera general o incluso mejor si está especialmente entrenado para detectar 
las clases que se están estudiando. De esta manera, se generarían buenas 
entradas para el clasificador, reduciendo el efecto que este tendría en el 
\textit{pipeline} final. En ese sentido, sería bueno poder realizar más 
experimentos con diferentes arquitecturas y modelos preentrenados que incluya 
la clase peces.
\\\\
De la misma manera, actualmente existen ciertos \textit{datasets} que son 
ampliamente utilizados para la investigación y creación de aplicaciones de 
DL, mientras que otros presentan pocas implementaciones, lo que hace más 
complicado la obtención de modelos preentrenados con las clases genéricas 
requeridas (por ejemplo, peces). En ese sentido, otro trabajo futuro 
consistiría en poder utilizar este etiquetador para crear un \textit{dataset} 
con la clase genérica de peces e incrementar una capa de atención a los modelos 
para poder analizar como se comporta el \textit{pipeline} con respecto al 
\textit{accuracy} parcial y final.   
\\\\
Por último, este trabajo se realizó con un \textit{dataset} de peces peruanos 
(en su mayoría) pero consistía de un bajo número de imágenes y consistían de 
una distribución de datos muy complejas. En ese sentido, el error y la 
precisión obtenidos también contemplaban \textit{bias} debido a estos factores. 
Es por ello que se recomendaría replicar las experimentaciones realizadas en 
el presente documento pero con un \textit{dataset} que incluya un número de 
especies e imágenes mayor para poder calcular como es que esto influye en la 
precisión, el costo de entrenamiento y si se necesita mejorar o incrementar 
algunas capas del modelo.
