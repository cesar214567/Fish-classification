
En los últimos años \textit{Machine Learning} (ML) ha ganado una mayor importancia por su gran capacidad de aprender patrones, realizar regresiones y especialmente debido al potencial que tienen los diferentes modelos de clasificación. En ese sentido \textit{Deep Learning}(DL), o aprendizaje profundo, sub-área de ML, ya ha demostrado ser mejor que el humano en diferentes aplicaciones, desde la detección, localización y clasificación de imágenes de rostros, números, lugares, e incluso fauna y flora. 
\newline
\newline
En el contexto de la fauna marina, existen aplicaciones a nivel global. Trabajos previos,  en este tópico, señalan que DL puede ser usado en la investigación de recursos pecuarios, obtención de conocimiento de las poblaciones, procesamiento de acuicultura, protección de especies en peligro de extinción y en el mantenimiento de la pesquería sustentable, entre otros \cite{10.1145/3419635.3419643, 10.1145/3325917.3325934,20.500.12724/11174,8371919}. \\\\
Dentro de estos trabajos, Nibha Manandhar y J. W. Burris \cite{10.1145/3325917.3325934}, además de Mejia y Rosales \cite{20.500.12724/11174} utilizaron \textit{transfer learning}(TL) para la creación del modelo, técnica inspirada del trabajo de Chen Guang, \textit{et al}.\cite{8371919}.
Por otra parte, también se viene investigando modelos para la clasificación en el fondo marino por Suxia Cui, \textit{et al}. \cite{Cui2020}, quienes usaron ImageNet\cite{ImageNet} como \textit{dataset} de imágenes para su entrenamiento. 
\newline
\newline
Siguiendo esta línea de investigación, sería conveniente comprobar los posibles usos que se le pueden dar en el contexto peruano tal como se hace en el mundo para optimizar o automatizar los procesos actuales de la clasificación. En el contexto actual, sin embargo, en el área de la clasificación de fauna marina peruana, únicamente Mejía y Rosales \cite{20.500.12724/11174} han realizado investigaciones sobre \textit{pipelines} de ML y visión computacional para esta área, analizando modelos con un gran costo computacional lo cual los hace poco factibles para su recreación en tiempo real.
\newline
\newline
Es por ello que el objetivo de este trabajo es implementar un pipeline que mejore la escalabilidad de estos modelos, evitando generar un costo computacional elevado y disminuyendo la precisión, para que pueda utilizarse en tiempo real en la clasificación de la fauna marina peruana. La estructura del documento se compone de: contexto y motivación, marco teórico, revisión de la literatura, metodología, experimentos y resultados, y por último, conclusiones y trabajos futuros.