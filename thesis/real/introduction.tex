
En los últimos años \textit{Machine Learning} (ML) ha ganado  
mayor importancia debido a su capacidad para aprender patrones, 
realizar regresiones y, especialmente, al potencial de los diferentes 
modelos de clasificación. En este contexto, \textit{Deep Learning}(DL), 
una subárea del ML, ha demostrado ser superior al humano en diversas 
aplicaciones, como la detección, localización y clasificación de imágenes 
de rostros, números, lugares e incluso fauna y flora.
\newline
\newline
En la actualidad existen muchas aplicaciones a nivel global para el contexto de 
la fauna marina. Trabajos previos señalan que DL puede utilizarse 
en la investigación de recursos pesqueros, la obtención de conocimiento 
sobre poblaciones, el procesamiento de acuicultura, la protección de 
especies en peligro de extinción y el mantenimiento de la pesquería 
sostenible, entre otros 
\cite{10.1145/3419635.3419643, 10.1145/3325917.3325934,20.500.12724/11174,8371919}.
\newline
\newline
Dentro de estos trabajos, Nibha Manandhar y J. W. Burris 
\cite{10.1145/3325917.3325934}, además de Mejia y Rosales 
\cite{20.500.12724/11174} utilizaron \textit{transfer learning}(TL) 
para crear sus modelos, una técnica inspirada en el trabajo de Chen Guang, 
\textit{et al}.\cite{8371919}. Además, se han investigado modelos para la 
clasificación en el fondo marino, como el estudio de Suxia Cui, 
\textit{et al}. \cite{Cui2020}, quienes emplearon \cite{ImageNet} como 
\textit{dataset} de imágenes para su entrenamiento. 
\newline
\newline
Siguiendo esta línea de investigación, sería conveniente explorar 
los posibles usos en el contexto peruano, similar a lo que se hace a nivel 
mundial, para optimizar o automatizar los procesos actuales de detección y 
clasificación, además de crear nuevos \textit{datasets} para futuros trabajos. 
Sin embargo, en el área de la clasificación de fauna marina 
peruana, Mejía y Rosales \cite{20.500.12724/11174} han sido los únicos en 
realizar investigaciones sobre \textit{pipelines} de ML y visión 
computacional, analizando modelos con un gran costo computacional lo cual 
los hace poco factibles para su recreación en tiempo real.
\newline
\newline
Por lo tanto, el objetivo de este trabajo es implementar un \textit{pipeline} 
que mejore la escalabilidad de estos modelos, evitando generar un costo 
computacional elevado y reduciendo la pérdida de precisión, para que pueda 
utilizarse en tiempo real en la detección, clasificación y etiquetado de la 
fauna marina peruana. 
La estructura del documento se compone de: contexto y motivación, marco 
teórico, revisión de la literatura, metodología, experimentos y resultados, 
y por último, conclusiones y trabajos futuros.