Machine Learning (ML) stands out as a fundamental tool for 
image detection and classification, however, training 
advanced models requires a large number of labeled images 
and significant computational capacity. This task is especially 
challenging in the Peruvian context due to the scarcity of labeled 
specimens. To address this problem, a Deep Learning (DL) pipeline 
was developed to create an automatic fish labeler in the Peruvian 
marine fauna. This pipeline uses a pretrained detector (YoloV5 and 
Unidet) and a classifier based on CNNs (EfficientNetB0). The 
selection of the classifier was based on an exhaustive analysis of 
various state-of-the-art models, considering the size in memory, 
the number of parameters and the precision obtained with the 
research data sets. The practical results showed a partial accuracy 
of the detector of 79.45\%, while the classifier reached 91.47\%, 
thus generating a final accuracy of 72.67\%. In addition, a minimum 
error of 22.54\% was achieved and a real-time application was developed 
that reached up to 8 fps, which allowed the image labeling task to be 
automated.