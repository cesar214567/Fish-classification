Como fue mencionado anteriormente, las investigaciones del estado del arte se han basado en proponer modelos de DL cada vez más precisos para la clasificación de imágenes. Lamentablemente aquello no implica que sean los mejores para su uso en aplicaciones en tiempo real, ya que aquellos modelos generan un gran costo computacional por imagen procesada, dejándolos no aptos para su uso en las empresas o para los usuarios no experimentados. Además que este tipo de modelos obliga al investigador a realizar trabajo manual al tener la necesidad de etiquetar de manera especial estas imágenes. \\\\
Por otra parte, cabe resaltar que hasta el momento no existen más investigaciones sobre el uso de esta tecnología en el contexto peruano, lo cual se puede evidenciar en el hecho que el trabajo de clasificación de peces se haga manualmente. Es por ello que actualmente existe una falta de \textit{datasets} de dominio público para investigaciones como esta. Son  aquellos tres motivos por los cuales se está realizando la presente investigación.  