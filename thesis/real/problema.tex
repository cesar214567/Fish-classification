Muchos investigadores en el campo del aprendizaje profundo (DL) 
han centrado sus esfuerzos en mejorar la precisión de sus modelos. Para 
lograrlo, han creado redes cada vez más complejas y con requisitos más 
exigentes para su entrenamiento. Dentro de ellos, los modelos de detección 
y clasificación actualmente requieren un \textit{dataset} etiquetado para su 
entrenamiento. Este proceso de etiquetado suele ser largo, costoso y precisa de  
especialistas, los cuales cuentan a su vez la posibilidad de cometer 
errores humanos. Todos estos problemas han dificultado la creación de 
\textit{datasets} peruanos para entrenar los modelos de vanguardia que pueden 
requerir miles de imágenes por especie, lo cual se puede evidenciar en la 
baja cantidad de investigaciones que se han llevado a cabo al respecto. 