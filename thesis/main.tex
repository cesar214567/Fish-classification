\documentclass{report}
\usepackage[utf8]{inputenc}
\usepackage[spanish]{babel}
\usepackage{import}
\usepackage{svg}
\usepackage{amsmath}
\usepackage{cite}
\usepackage[colorinlistoftodos]{todonotes}
\usepackage[most]{tcolorbox}
\usepackage{hyperref}
\usepackage{rotating}


\title{\textit{Machine Learning} Pipeline para el etiquetado 
automático en imágenes de especies 
de peces peruanos}
\author{cesar.madera@utec.edu.pe}
\date{Noviembre 2021}
\usepackage{graphicx}
\usepackage{subcaption}
\usepackage{apacite}
\usepackage{csquotes}

%\usepackage{natbib}
% \bibliographystyle{abbrvnat}
%\bibliographystyle{elsarticle-num-names}
\usepackage{graphicx}

%no corte palabras
\tolerance=1
\emergencystretch=\maxdimen
\hyphenpenalty=10000
\hbadness=10000


\begin{document}

\begin{titlepage}
    \begin{center}
        \Large
        \textbf{UNIVERSIDAD DE INGENIERÍA Y TECNOLOGÍA}
        \vspace*{1cm}

        \large
        \textbf{CARRERA DE CIENCIA DE LA COMPUTACIÓN}
        \vspace*{1cm}

        \begin{figure}[htbp]
            \centering
            \includegraphics[width=6.5cm,height=\textheight,keepaspectratio]{images/logo}
        \end{figure}


        \LARGE
        \textbf{Pipeline para el etiquetado 
            automático en imágenes de especies 
            de peces peruanos}

        \vspace{1.0cm}
        \Large

        %\vspace{1.5cm}

        \textbf{AUTOR}
        \vspace{0.5cm}
        \\Cesar Antonio Madera Garcés
        \\cesar.madera@utec.edu.pe

        \textbf{ASESOR}
        \vspace{0.5cm}
        \\Cristian López
        \\clopezd@utec.edu.pe
        \vfill
        \Large

        Lima - Perú
        \\
        2022

    \end{center}
\end{titlepage}


\chapter*{Resumen}
\import{}{real/resumen.tex}

\chapter*{Abstract}
\import{}{real/abstract.tex}

\tableofcontents

%---------- Capítulo  Contexto y Motivación -------
\chapter{Contexto y Motivación}

\section{Introducción}
\import{}{real/introduction.tex}

\section{Descripción del problema}
\import{}{real/problema.tex}

\section{Justificación}
\import{}{real/justificacion.tex}

\section{Objetivos}
\import{}{real/objetivos.tex}

\section{Aportes}
\import{}{real/aportes.tex}


%--------------------------------------------------

%---------- Capítulo  Marco Teórico ---------------
\chapter{Marco Teórico}
\import{}{real/prevKnowledge.tex}
%--------------------------------------------------

%--------------------------------------------------
%---------- Capítulo  Revisión Lit. ---------------
\chapter{Revisión de la Literatura}
\import{}{real/stateOfTheArt.tex}
%\hline
%\LARGE
%Hasta aquí la evaluación parcial por parte de un jurado invitado

%\hline
%--------------------------------------------------

%---------- Capítulo  Metodología -----------------
\chapter{Metodología}
\import{}{real/proposal.tex}

%\section{Alcances y Limitaciones}
%\import{}{template/alcances.tex}


%--------------------------------------------------

%---------- Capítulo  Experimentación -----------------

\chapter{Experimentaciones y Resultados}
\import{}{real/experiments.tex}

%--------------------------------------------------

%---------- Capítulo  Conclusiones -----------------

\chapter{Conclusiones y Trabajos Futuros}
%\import{}{template/Prevconclusions.tex}

\import{}{real/conclusions.tex}


%\hline
%\LARGE
%Hasta aquí la evaluación final  por parte de un jurado invitado





%\hline


%\bibliographystyle{plain}

\nocite{zobel}
\nocite{swales}

\bibliographystyle{apacite}
\bibliography{references}

\newpage
\appendix

\chapter{Experimentación con distribución de datos diferentes}
\import{}{real/anexos.tex}



\end{document}
