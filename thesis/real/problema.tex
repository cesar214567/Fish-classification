Muchos investigadores en la sub-area de DL se han enfocado en mejorar la precisión de sus modelos, creando redes con mayor complejidad y mayores requerimientos específicos para su creación y entrenamiento. Por otro lado, no se ha investigado acerca del uso en tiempo real de estas redes en un contexto de la detección y localización de la fauna marina peruana en imágenes o video, sobretodo por la alta necesidad de ejemplares dentro de una base de datos para entrenar los modelos del estado del arte. En ese sentido podríamos acotar el problema a que no existe algún \textit{pipeline} o modelo que pueda ser entrenado en base a ejemplares peruanos debido a la falta de ellos y a la complejidad de los modelos o \textit{pipelines} actuales. 