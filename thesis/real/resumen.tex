\textit{Machine Learning} (ML) se destaca como una herramienta 
fundamental para la detección y clasificación de imágenes, 
sin embargo, el entrenamiento de modelos avanzados requiere 
una gran cantidad de imágenes etiquetadas y una capacidad 
computacional significativa. Esta tarea resulta especialmente 
desafiante en el contexto peruano debido a la escasez de 
ejemplares etiquetados. Para abordar esta problemática, se 
desarrolló un \textit{pipeline} de \textit{Deep Learning} (DL) para crear un 
etiquetador automático de peces en la fauna marina peruana. 
Este \textit{pipeline} utiliza un detector preentrenado (YoloV5 y Unidet) 
y un clasificador basado en CNNs (EfficientNetB0). La selección 
del clasificador se basó en un análisis exhaustivo de diversos 
modelos del estado del arte, considerando el tamaño en memoria, 
el número de parámetros y la precisión obtenida con los conjuntos 
de datos de la investigación. Los resultados prácticos mostraron 
un accuracy parcial del detector del 79.45\%, mientras que el 
clasificador alcanzó un 91.47\%, generando así un accuracy final 
del 72.67\%. Además, se logró un error mínimo del 22.54\% y se 
desarrolló una aplicación en tiempo real que alcanzó hasta 8 fps, 
lo que permitió automatizar la tarea de etiquetado de imágenes.