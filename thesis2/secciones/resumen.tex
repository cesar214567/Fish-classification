\chapter*{\center \Large \vspace{-4.5cm} RESUMEN}
\addcontentsline{toc}{section}{\bfseries RESUMEN}
\markboth{RESUMEN}{RESUMEN} 

\textit{Machine Learning} (ML) se destaca como una herramienta 
fundamental para la detección y clasificación de imágenes. 
Sin embargo, el entrenamiento de modelos avanzados requiere 
una gran cantidad de imágenes etiquetadas y una capacidad 
computacional significativa. Esta tarea resulta especialmente 
desafiante en el contexto de la fauna marina peruana, debido 
a la escasez de conjuntos de datos etiquetados. Para abordar esta problemática, 
se desarrolló un etiquetador automático de peces basado en un 
\textit{pipeline} de \textit{Deep Learning} (DL). 
Este \textit{pipeline} utiliza un detector preentrenado (YoloV5 y Unidet) 
y una red EfficientNetB0, clasificador basado en 
\textit{Convolutional Neural Networks} (CNN's). 
La selección del clasificador se basó en un análisis exhaustivo de diversos 
modelos del estado del arte, considerando el tamaño en memoria, 
el número de parámetros y la precisión obtenida con los conjuntos 
de datos de la investigación. Los resultados prácticos mostraron 
una precisión parcial del detector del 79.45\%, mientras que el 
clasificador alcanzó un 91.47\%, generando así una precisión final 
del 72.67\%. Además, se logró un error mínimo del 22.54\% y se 
desarrolló una aplicación en tiempo real que alcanzó hasta 8 fps, 
lo que permitió automatizar la tarea de etiquetado de imágenes.


\noindent \textbf{Palabras clave:}\\
\noindent Machine Learning; Deep Learning; Detectores; CNN's; Pipelines; Etiquetado Automático;