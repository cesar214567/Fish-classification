A lo largo de los tiempos, el error humano y la falta de la automatización ha sido uno de los problemas más grandes de las instituciones y usuarios inexpertos en general. A consecuencia de esto, la creación y uso de aplicaciones con inteligencia artificial para la detección y clasificación de imágenes ha ido aumentando conforme al tiempo para disminuir el posible error humano. En el contexto peruano, esta tarea  nunca ha sido automatizada, generando también una falta de \textit{datasets} con imágenes etiquetadas. Actualmente, los modelos de \textit{Deep Learning}(DL) son los más usados generando una necesidad de este tipo de \textit{datasets} . Es por ello que en el siguiente trabajo se propuso, implementó y experimentó con un \textit{pipeline} de DL para la detección y clasificación de peces dentro de la fauna marina peruana, finalmente escogiendo una configuración que redujo la pérdida en el \textit{accuracy} y el tiempo de entrenamiento. Para ello, se evaluaron diferentes modelos del estado del arte en base al tamaño en memoria, el numero de parámetros y la precisión obtenida con el \textit{dataset} de ImageNet como métricas teóricas para su comparación, para luego aplicar \textit{transfer learning} sobre ellas para adaptarlas a nuestra aplicación y obtener un balance entre precisión-costo computacional para finalmente obtener la mas eficiente.  \\