\section{Conclusiones}
En el presente trabajo se desarrolló un \textit{pipeline} para el
etiquetado de imágenes de peces peruanos, empleando técnicas de detección y 
clasificación en un entorno real. Para ello, se recolectó un 
\textit{dataset} de 5 mil imágenes aproximadamente a través de la plataforma 
Kaggle, conteniendo varios peces dentro de cada imagen, el cual fue 
recortado y clasificado por un experto. 
A través de la experimentación, se comprobó que un modelo 
ligero pero al mismo tiempo robusto como lo es EfficientNetB0 obtuvo una 
precisión similar (entre 76\% y 92\%) al ser usado con diferentes detectores 
dentro del \textit{pipeline}. 
De parte del detector, se pudo comprobar que un modelo entrenado (como lo es YoloV5) 
logra una mejor precisión inclusive comparándolo con una arquitectura más compleja 
(como lo es UniDet), teniendo una diferencia de 79.5\% y 40.15\%. Por el contrario, 
se obtuvo un menor error en la detección hecha por Unidet en general, siendo esta de 
22.54\%. 
\newline
\newline
Considerando los resultados obtenidos podemos afirmar que el performance y 
la precisión del \textit{pipeline} dependerá del modelo preentrenado para el detector, 
ya que el clasificador logrará clasificar correctamente (en su mayoría de veces) para lo 
que fue entrenado. 
En base a ellos, se logró desarrollar una aplicación en tiempo real, el cual 
permitió automatizar el proceso de etiquetado para imágenes y videos.

\section{Trabajos futuros}

Algunas de las limitaciones que se tiene con este modelo consiste en que el 
\textit{pipeline} depende casi completamente del detector escogido, de tal 
manera que se necesita poder obtener uno bastante robusto y entrenado de 
manera general o incluso mejor si está especialmente entrenado para detectar 
las clases que se están estudiando. De esta manera, se generarían buenas 
entradas para el clasificador, reduciendo el efecto que este tendría en el 
\textit{pipeline} final. En ese sentido, sería bueno poder realizar más 
experimentos con diferentes arquitecturas y modelos preentrenados que incluya 
la clase peces.
\\\\
De la misma manera, actualmente existen ciertos \textit{datasets} que son 
ampliamente utilizados para la investigación y creación de aplicaciones de 
DL, mientras que otros presentan pocas implementaciones, lo que hace más 
complicado la obtención de modelos preentrenados con las clases genéricas 
requeridas (por ejemplo, peces). En ese sentido, otro trabajo futuro 
consistiría en poder utilizar este etiquetador para crear un \textit{dataset} 
con la clase genérica de peces e incrementar una capa de atención a los modelos 
para poder analizar como se comporta el \textit{pipeline} con respecto al 
\textit{accuracy} parcial y final.   
\\\\
Por último, este trabajo se realizó con un \textit{dataset} de peces peruanos 
(en su mayoría) pero consistía de un bajo número de imágenes y consistían de 
una distribución de datos muy complejas. En ese sentido, el error y la 
precisión obtenido también contemplaban \textit{bias} debido a estos factores. 
Es por ello que se recomendaría replicar las experimentaciones realizadas en 
el presente documento pero con un \textit{dataset} que incluya un número de 
especies e imágenes mayor para poder calcular como es que esto influye en la 
precisión, el costo de entrenamiento y si se necesita mejorar o incrementar 
algunas capas del modelo.
