\chapter*{\center \Large CONCLUSIONES} 
\addcontentsline{toc}{section}{\bfseries CONCLUSIONES} 
\markboth{CONCLUSIONES}{CONCLUSIONES} 

En el presente trabajo se desarrolló un \textit{pipeline} para el
etiquetado de imágenes de peces peruanos, empleando técnicas de detección y 
clasificación en un entorno real. Para ello, se recolectó un 
\textit{dataset} de 5 mil imágenes aproximadamente a través de la plataforma 
Kaggle, conteniendo varios peces dentro de cada imagen, el cual fue 
recortado y clasificado por un experto. 
A través de la experimentación, se comprobó que un modelo 
ligero pero al mismo tiempo robusto como lo es EfficientNetB0 obtuvo una 
precisión similar (entre 76\% y 92\%) al ser usado con diferentes detectores 
dentro del \textit{pipeline}. 
De parte del detector, se pudo comprobar que un modelo entrenado (como lo es YoloV5) 
logra una mejor precisión inclusive comparándolo con una arquitectura más compleja 
(como lo es UniDet), teniendo una diferencia de 79.5\% y 40.15\%. Por el contrario, 
se obtuvo un menor error en la detección hecha por Unidet en general, siendo esta de 
22.54\%. 
\newline
\newline
Considerando los resultados obtenidos podemos afirmar que el performance y 
la precisión del \textit{pipeline} dependerá del modelo preentrenado para el detector, 
ya que el clasificador logrará clasificar correctamente (en su mayoría de veces) para lo 
que fue entrenado. En base a ellos, se logró desarrollar una aplicación en tiempo real, 
el cual permitió automatizar el proceso de etiquetado para imágenes y videos.

Cabe recalcar que la intención de este \textit{pipeline} no es reemplazar o mejorar al 
sistema basado únicamente en YOLO, sino ser un sustituto real en casos en donde 
se tenga un \textit{dataset} sin etiquetar y se quiera generar una solución a un 
problema de localización y clasificación de objetos. 


