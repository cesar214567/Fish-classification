\chapter{INTRODUCCI\'ON}

En los últimos años \textit{Machine Learning} (ML) ha ganado  
mayor importancia debido a su capacidad para aprender patrones, 
realizar regresiones y, especialmente, al potencial de los diferentes 
modelos de detección y clasificación de objetos. En este contexto, \textit{Deep Learning} (DL), 
una subárea del ML, ha demostrado ser superior al humano en diversas 
aplicaciones como el reconocimiento de rostros, números, lugares e incluso fauna y flora.
\newline
\newline
Actualmente, existen muchas aplicaciones a nivel global en el contexto de la fauna marina. 
Por ejemplo, algunos trabajos previos han utilizado DL para incrementar la investigación de 
recursos pesqueros, obtener conocimiento sobre poblaciones, procesar la acuicultura, proteger 
especies en peligro de extinción y mantener la pesquería sostenible, entre otros
\cite{10.1145/3419635.3419643, 10.1145/3325917.3325934,20.500.12724/11174,8371919}.
\newline
\newline
Dentro de estos trabajos, Nibha Manandhar y J. W. Burris , además de 
Mejia y Rosales utilizaron \textit{Transfer Learning} (TL)para crear sus modelos, 
una técnica inspirada en el trabajo de Chen Guang, 
\textit{et al}. Además, se han investigado modelos para la 
clasificación en el fondo marino, como el estudio de Suxia Cui, 
\textit{et al}. \cite{Cui2020}, quienes emplearon \textit{ImageNet} \cite{ImageNet} como 
\textit{dataset} de imágenes para su entrenamiento. 
\newline
\newline
Siguiendo el camino trazado por las investigaciones anteriores, sería conveniente explorar 
los posibles usos en el contexto peruano, similar a lo que se hace a nivel 
mundial, para optimizar o automatizar los procesos actuales de detección y 
clasificación. Sin embargo, en el área de la clasificación de fauna marina 
peruana, Mejía y Rosales han sido los únicos en 
realizar investigaciones sobre \textit{pipelines} de ML y visión 
computacional hasta es de nuestro conocimiento. Ellos analizaron modelos con 
un gran costo computacional en tiempo de inferencia lo cual los hace poco 
factibles para su recreación en tiempo real. 
\newline
\newline
Además del uso de modelos complejos en el estado del arte, existe una escazes 
de imágenes que incluyan múltiples especies dentro de la fauna marina peruana. 
Esto debido a que es necesario un proceso de etiquetado, que consiste en la creación 
de recuadros (\textit{bounding boxes}) conteniendo al pez y su respectiva clase.
Esta tarea debe ser realizada por personas expertas que podrían o no tener experiencia 
tecnológica, lo que dificulta aún más poder realizar avances en esta área o entrenar 
los modelos de vanguardia para la fauna peruana. 
\newline
\newline
Por lo tanto, en este trabajo se propone desarrollar un \textit{pipeline} 
para el etiquetado automático en imágenes de especies marinas peruanas. 
La finalidad de este \textit{pipeline} es evitar la necesidad de un 
\textit{dataset} etiquetado, generar una mayor escalabilidad de estos 
modelos, generar un menor costo computacional y reducir la pérdida de 
precisión. Estas mejoras ayudarán a facilitar el uso a personas sin experiencia 
en tecnología, permitiendoles ejecutar el \textit{pipeline} en tiempo real 
en la detección, clasificación y etiquetado de la fauna marina peruana. 
\newline
\newline
La estructura del documento se compone de: contexto y motivación, marco 
teórico, revisión de la literatura, metodología, experimentos y resultados, 
y por último, conclusiones y trabajos futuros.


\section{Descripción del problema}

Muchos investigadores en el campo del aprendizaje profundo (DL) 
han centrado sus esfuerzos en mejorar la precisión de sus modelos. Para 
lograrlo, han creado redes cada vez más complejas y con requisitos más 
exigentes para su entrenamiento. Dentro de ellos, los modelos de detección 
y clasificación actualmente requieren un \textit{dataset} etiquetado para su 
entrenamiento. Este proceso de etiquetado suele ser largo, costoso y precisa de  
especialistas, los cuales cuentan a su vez la posibilidad de cometer 
errores humanos. Todos estos problemas han dificultado la creación de 
\textit{datasets} peruanos para entrenar los modelos de vanguardia que pueden 
requerir miles de imágenes por especie, lo cual se puede evidenciar en la 
baja cantidad de investigaciones que se han llevado a cabo al respecto. 

\section{Justificación}

La creación de un \textit{pipeline} que permita generar un etiquetado 
automático en la fauna marina peruana no solo facilitaría la creación 
nuevos \textit{datasets} relacionados con la fauna marina peruana, sino 
también la clasificación de objetos en general, los cuales 
podrían ser disponibilizados a los investigadores para futuros trabajos. 
Además, este \textit{pipeline} tendría un bajo costo computacional, tanto 
para su entrenamiento como para su uso en entornos reales, aprovechando el 
conocimiento previo de redes ya entrenadas y aportando una aplicación para la 
detección y clasificación de peces peruanos en tiempo real.

\section{Objetivos}

\begin{itemize}
  \item { Objetivo principal: 
      \begin{itemize}
          \item Proponer, desarrollar y probar un 
              \textit{pipeline} de DL para el etiquetado de imágenes complejas 
              con múltiples especies de peces en la fauna marina peruana.
      \end{itemize}
   }
   \item { Objetivos secundarios:
      \begin{itemize}
          \item Buscar y unificar \textit{datasets} de imágenes de la fauna marina 
          peruana.
          \item Realizar un análisis comparativo del \textit{pipeline} propuesto para 
              distintos detectores y clasificadores del estado del arte. 
          \item Automatizar el proceso de etiquetado a través de un aplicativo.
      \end{itemize}
      }
\end{itemize}

\section{Aportes}

Este trabajo no solo comparará diferentes arquitecturas 
de redes neuronales convolucionales (CNN) actuales, lo que facilitará la 
elección de un modelo para futuros investigadores, sino también proporcionará 
una solución para agregar especificidad a las clases de los \textit{datasets} 
de los modelos ya preentrenados. Esta estrategia evitará la necesidad de 
entrenar modelos más complejos y permitirá etiquetar un conjunto de datos 
para futuros trabajos.
