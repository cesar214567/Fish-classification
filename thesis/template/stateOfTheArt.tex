\todo{CS4002}
\todo[color=blue!40]{CS4003}
\todo[color=green!40]{CS4004}
Una revisión de la literatura es un resumen crítico y analítico, y una síntesis del conocimiento actual de un tema. Una revisión de la literatura es mucho más que una lista de revisiones separadas de artículos y libros debe comparar y relacionar diferentes teorías, hallazgos, etc., en lugar de resumirlos individualmente. También debe tener un enfoque o tema particular para organizar la revisión.  Recuerde que no tiene por qué ser un relato exhaustivo de todo lo publicado sobre el tema, sino debería discutir toda la literatura académica más significativa e importante para ese enfoque.

El estado del arte o revisión bibliográfica permite pocisionar nuestro proyecto dentro de los trabajos ya existentes en la literatura científica. En la revisión bibliográfica se deben revisar solamente documentos pertenecientes a la literatura primaria y secundaria, evitando la literatura terciaria y gris y la literatura no científica. Es importante definir el formato de la citaciones (e.g., APA) y las forma correcta de hacerlo.

Existen varias técnicas para construir esta parte de un proyecto de investigación. Podemos utilizar una técnica poco formal, como la Revisión Empírica o Narrativa o podemos utilizar una técnica mas estructurada como la Revisión Sistemática~\cite{moreno2018revisiones}.

La revisión de la literatura debe concluir con un resumen y una pequeña discusión.
