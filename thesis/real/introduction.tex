
En los últimos años \textit{Machine Learning} (ML) ha ganado  
mayor importancia debido a su capacidad para aprender patrones, 
realizar regresiones y, especialmente, al potencial de los diferentes 
modelos de detección y clasificación de objetos. En este contexto, \textit{Deep Learning}(DL), 
una subárea del ML, ha demostrado ser superior al humano en diversas 
aplicaciones como el reconocimiento de rostros, números, lugares e incluso fauna y flora.
\newline
\newline
En la actualidad existen muchas aplicaciones a nivel global para el contexto de 
la fauna marina. Trabajos previos señalan que DL puede utilizarse 
en la investigación de recursos pesqueros, la obtención de conocimiento 
sobre poblaciones, el procesamiento de acuicultura, la protección de 
especies en peligro de extinción y el mantenimiento de la pesquería 
sostenible, entre otros 
\cite{10.1145/3419635.3419643, 10.1145/3325917.3325934,20.500.12724/11174,8371919}.
\newline
\newline
Dentro de estos trabajos, Nibha Manandhar y J. W. Burris 
\cite{10.1145/3325917.3325934}, además de Mejia y Rosales 
\cite{20.500.12724/11174} utilizaron \textit{transfer learning}(TL) 
para crear sus modelos, una técnica inspirada en el trabajo de Chen Guang, 
\textit{et al}.\cite{8371919}. Además, se han investigado modelos para la 
clasificación en el fondo marino, como el estudio de Suxia Cui, 
\textit{et al}. \cite{Cui2020}, quienes emplearon \textit{ImageNet}\cite{ImageNet} como 
\textit{dataset} de imágenes para su entrenamiento. 
\newline
\newline
Siguiendo esta línea de investigación, sería conveniente explorar 
los posibles usos en el contexto peruano, similar a lo que se hace a nivel 
mundial, para optimizar o automatizar los procesos actuales de detección y 
clasificación. Sin embargo, en el área de la clasificación de fauna marina 
peruana, Mejía y Rosales \cite{20.500.12724/11174} han sido los únicos en 
realizar investigaciones sobre \textit{pipelines} de ML y visión 
computacional hasta donde se ha investigado. Ellos analizaron modelos con 
un gran costo computacional en tiempo de inferencia lo cual los hace poco 
factibles para su recreación en tiempo real. Además del uso de estos 
modelos complejos en el estado del arte, que requieren grandes cantidades 
de imágenes, existe una falta de \textit{datasets} que incluyan especies 
dentro de la fauna marina peruana, lo que dificulta aún más poder realizar 
avances en esta área o entrenar los modelos de vanguardia para la fauna 
peruana.
\newline
\newline
Por lo tanto, en este trabajo se propone desarrollar un \textit{pipeline} 
para el etiquetado automático en imágenes de especies marinas peruanas 
logrando una escalabilidad de estos modelos, generando un menor costo 
computacional y reduciendo la pérdida de precisión, para que pueda 
utilizarse en tiempo real en la detección, clasificación y etiquetado de 
la fauna marina peruana. 
\newline
\newline
La estructura del documento se compone de: contexto y motivación, marco 
teórico, revisión de la literatura, metodología, experimentos y resultados, 
y por último, conclusiones y trabajos futuros.